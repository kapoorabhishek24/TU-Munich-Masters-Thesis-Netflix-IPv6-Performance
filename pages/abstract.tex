\chapter{\abstractname}

We are analyzing a 22-months long (since July 2016) Netflix dataset to uncover the
performance of the content delivery network over IPv6. As IPv4 address space is
getting exhausted due to increase in the number of connected devices, IPv6 was proposed
to counter this problem, but after two decades of its releasing, IPv6 is still not fully
adopted by the network and content providers. The main aim of this study was to
look into the performance over IPv6 with respect to different aspects. The Netflix
dataset that we analyzed was collected using 100 SamKnows probes connected to
dual-stacked networks representing 66 different origin ASes. The results show that a
Happy Eyeballs (HE) race during initial TCP connection establishment leads to a strong
(around 93\%) preference over IPv6. However, even though clients prefer streaming
videos over IPv6, worse performance over IPv6 than over IPv4 was observed, whereby
consistent higher TCP connection establishment times and pre-buffering duration (40ms or more) 
were witnessed over IPv6. Similarly, consistent lower achieved throughput over IPv6 was also observed. 
Less than 10\% stall rates over both address families were
observed. ISP content caches do have an impact on the latency and throughput, and as
per the results it was observed that content caches lead to reduced TCP connect times
and Pre-buffering duration. Also, content caches achieved higher throughput (around
66\% of the times) over both IPv4 and IPv6. We also analyze the Speedtest dataset,
which is speedtest measurements to Measurement Lab (M-Lab) servers, collected using
the same probes. The comparison between M-Lab and Netflix speedtest reveals that the
speedtest is better towards M-Lab servers than towards Netflix OCA servers. We did
find out that the path length (TTL) to M-Lab servers is comparatively less than the TTL
to Netflix OCA servers, indicating that shorter path lengths correlate with a higher speed
for M-Lab. From the traceroute data for Netflix, it was observed that content caches
had reduced path lengths, TCP connect times and Pre-buffering durations, and can be
reached within 5 hops and 21ms. The results from the analysis indicates good ongoing
adoption of the IPv6 protocol.


 


